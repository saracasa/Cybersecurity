\documentclass[12pt,a4paper]{tau-class/tau}
\usepackage[italian]{babel}


\journalname{Cybersecurity}
\title{Dati Sanitari Sintetici: Compromesso tra Privacy e Utilità nella Ricerca Medica}

\begin{document}

\begin{titlepage}
  \centering
  \vspace*{2cm}

  {\Huge \textbf{\journalname}}\\[1.5cm]
  {\LARGE \textbf{\thetitle}}\\[1.5cm]

  \textbf{Autori:} \\
  Sara Casadio \\
  Noemi Ferrara \\
  Giorgia Pirelli 000117616\\

  \vspace{0.5cm}
  {\large Anno accademico 2025/2026}\\[0.5cm]
  % Date
  {\large 26 Novembre, 2025}\\[0.5cm]

  \vfill
\end{titlepage}

%----------------------------------------------------------

\section{Introduzione}
Negli ultimi anni, l'Intelligenza Artificiale e il Machine Learning hanno
aperto nuove prospettive nella medicina, consentendo diagnosi più precise e
terapie personalizzate grazie all'analisi di grandi dataset clinici. Tuttavia,
l'utilizzo di dati reali dei pazienti comporta rischi significativi legati alla
privacy e alla sicurezza, oltre alla necessità di rispettare normative rigorose
come il GDPR europeo e l'HIPAA statunitense. La sensibilità dei dati medici è
confermata anche dal loro valore sul mercato illecito: una cartella clinica può
arrivare a costare fino a 250 dollari, rendendoli un obiettivo attraente per i
cybercriminali.

I dati sintetici si propongono come un'alternativa innovativa. Generati tramite
tecniche come le Generative Adversarial Networks (GAN) o i modelli di
diffusione, questi dati replicano le caratteristiche statistiche dei dataset
reali senza contenere informazioni identificabili sui pazienti. Questo
approccio offre la possibilità di sviluppare modelli predittivi efficaci e di
condurre analisi approfondite riducendo i rischi etici e legali.

Tuttavia, rimane una questione centrale: è possibile produrre dati sintetici
che siano al contempo sicuri e sufficientemente utili? Una protezione della
privacy troppo restrittiva può compromettere l'utilità dei dati, mentre dati
sintetici molto realistici possono essere vulnerabili a attacchi come il
\textit{membership inference}, che rivela se un individuo è presente nel
dataset originale, o il \textit{re-identification}, che collega dati sintetici
a persone reali.

Diversi lavori precedenti hanno esplorato metodi di generazione e metriche di
valutazione dei dati sintetici. Ad esempio, Figueira et al. descrivono vari
approcci di sintesi, mentre Hernandez et al. confrontano strumenti di
valutazione per determinarne l'efficacia. Il progetto si concentra sul
bilanciamento tra privacy e utilità, analizzando quantitativamente come questi
due aspetti influenzino le prestazioni dei dati sintetici in contesti sanitari. %aggiungere riferimenti ad altri studi

%strumenti utilizzati

\subsection{Domande di ricerca}
Questo studio si propone di rispondere alle seguenti domande:
\begin{itemize}
  \item \textbf{RQ1:} Quali modelli mantengono le prestazioni più elevate quando addestrati su dati sintetici rispetto alla baseline con dati reali?
  \item \textbf{RQ2:} È possibile conciliare privacy e utilità in un dataset sintetico, o devono essere accettati compromessi significativi?
\end{itemize}

Per rispondere a queste domande, viene utilizzato l'UCI Diabetes Dataset, un
dataset pubblico contenente 768 pazienti con variabili mediche predittive e
diagnosi di diabete. 

L'analisi si sviluppa attraverso quattro fasi. Innanzitutto, vengono generati
dataset sintetici con diversi livelli di protezione della privacy: nessuna
privacy, privacy moderata e privacy forte. Successivamente, viene valutata la
somiglianza statistica confrontando distribuzioni e matrici di correlazione tra
dati reali e sintetici mediante test statistici standard. Nella terza fase,
viene verificata l'utilità dei dati sintetici addestrando modelli predittivi
che imparano a diagnosticare il diabete valutandoli su dati reali attraverso le metriche Accuracy,
Precision, Recall, F1-Score e ROC-AUC. Viene poi implementato un attacco di
\textit{membership inference} per testare la resistenza dei dati sintetici e
individuare eventuali fughe di informazioni. Infine, il compromesso tra privacy
e utilità viene visualizzato attraverso grafici che mostrano come le diverse
configurazioni influenzino le prestazioni, permettendo di identificare il punto
di equilibrio ottimale.

\section{Generezione dati sintetici}

%----------------------------------------------------------

\printbibliography

%----------------------------------------------------------

\end{document}